\PassOptionsToPackage{unicode=true}{hyperref} % options for packages loaded elsewhere
\PassOptionsToPackage{hyphens}{url}
%
\documentclass[]{article}
\usepackage{lmodern}
\usepackage{amssymb,amsmath}
\usepackage{ifxetex,ifluatex}
\usepackage{fixltx2e} % provides \textsubscript
\ifnum 0\ifxetex 1\fi\ifluatex 1\fi=0 % if pdftex
  \usepackage[T1]{fontenc}
  \usepackage[utf8]{inputenc}
  \usepackage{textcomp} % provides euro and other symbols
\else % if luatex or xelatex
  \usepackage{unicode-math}
  \defaultfontfeatures{Ligatures=TeX,Scale=MatchLowercase}
\fi
% use upquote if available, for straight quotes in verbatim environments
\IfFileExists{upquote.sty}{\usepackage{upquote}}{}
% use microtype if available
\IfFileExists{microtype.sty}{%
\usepackage[]{microtype}
\UseMicrotypeSet[protrusion]{basicmath} % disable protrusion for tt fonts
}{}
\IfFileExists{parskip.sty}{%
\usepackage{parskip}
}{% else
\setlength{\parindent}{0pt}
\setlength{\parskip}{6pt plus 2pt minus 1pt}
}
\usepackage{hyperref}
\hypersetup{
            pdfborder={0 0 0},
            breaklinks=true}
\urlstyle{same}  % don't use monospace font for urls
\setlength{\emergencystretch}{3em}  % prevent overfull lines
\providecommand{\tightlist}{%
  \setlength{\itemsep}{0pt}\setlength{\parskip}{0pt}}
\setcounter{secnumdepth}{0}
% Redefines (sub)paragraphs to behave more like sections
\ifx\paragraph\undefined\else
\let\oldparagraph\paragraph
\renewcommand{\paragraph}[1]{\oldparagraph{#1}\mbox{}}
\fi
\ifx\subparagraph\undefined\else
\let\oldsubparagraph\subparagraph
\renewcommand{\subparagraph}[1]{\oldsubparagraph{#1}\mbox{}}
\fi

% set default figure placement to htbp
\makeatletter
\def\fps@figure{htbp}
\makeatother


\date{}

\begin{document}

\hypertarget{resptech-wallpaper-public-license}{%
\section{RespTech Wallpaper Public
License}\label{resptech-wallpaper-public-license}}

This is the version 1 of the RespTech Wallpaper Public License, written
by Nguyen Hoang Duong, published as 42tm, 24 July 2018.

You can copy this license, and you can modify it under the condition
that you must not call your modified version the ``RespTech Wallpaper
Public License''.

\hypertarget{preamble}{%
\subsection{PREAMBLE}\label{preamble}}

This License is for any work that is intended to be a wallpaper that
everyone can use. We (42tm) created this License so that both the
author(s) and people who use the wallpaper can benefit. We hope to
encourage the creation of open, public art works for the benefit of the
large public.

\hypertarget{terms-and-conditions}{%
\subsection{TERMS AND CONDITIONS}\label{terms-and-conditions}}

\hypertarget{definitions}{%
\subsubsection{Definitions}\label{definitions}}

\begin{enumerate}
\def\labelenumi{\arabic{enumi}.}
\item
  ``This license'' means the terms and conditions described from the
  section ``Use of the wallpaper'' to section ``Limitations'' of this
  document.
\item
  ``RTWPL'' is the acronym for ``RespTech Wallpaper Public License'',
  which has the same definition as ``this license'' (above).
\item
  ``RTWPL-licensed'' means (something) licensed under the RespTech
  Wallpaper Public License.
\item
  ``Wallpaper'' means a form of visual art that is intended to be set as
  background on any digital device that supports background image
  (e.g.~phone, tablet, PC,\ldots{}). Wallpaper, in this context, can be
  of any type, as long as the device supports it (usually raster
  graphic).
\item
  ``Background'' has the same meaning as ``Wallpaper'', defined above.
\item
  ``User'' means any person who wish to use or is using the provided
  RTWPL-licensed Wallpaper.
\item
  ``Author'' means the author of the wallpaper, which might be an
  individual or a group of individuals. We define ``author'' as such to
  avoid having to write something like ``author(s)'' which can break the
  flow of the license text.
\item
  ``Provider'' means any individual or group of individuals who provide
  any RTWPL-licensed work.
\item
  ``Digital device'' means any digital device that can have wallpaper. A
  digital device may have many wallpapers set for different place
  (e.g.~the iPhone has wallpaper for the lock screen and has another
  wallpaper for the home screen.).
\item
  ``Original wallpaper'' means the wallpaper that someone else copied
  and modified the copy.
\item
  ``Modified wallpaper'' means the wallpaper that is essentially a
  modified version of another wallpaper (i.e.~the original wallpaper).
\item
  To use the wallpaper means to set the wallpaper as wallpaper on the
  digital device.
\item
  To modify the wallpaper means to edit the content of the wallpaper, or
  more specifically, making the wallpaper visually different from the
  original wallpaper.
\item
  To redistribute the wallpaper means to make a copy of a RTWPL-licensed
  wallpaper available somewhere public, such as on a web page, or in a
  museum.
\end{enumerate}

\hypertarget{use-of-the-wallpaper}{%
\subsubsection{Use of the wallpaper}\label{use-of-the-wallpaper}}

The wallpaper is provided free of charge, for anyone to use. Anyone can
set this wallpaper as wallpaper on any digital device as long as the
device supports the file type of the wallpaper. The author makes no
guarantee that the wallpaper can be set as wallpaper on every device. In
the event that you cannot set the wallpaper as background because your
device does not support the file type of the provided wallpaper, the
best you can do is asking the author to make a version with the file
type that your digital device supports.

\hypertarget{modifying-the-wallpaper}{%
\subsubsection{Modifying the wallpaper}\label{modifying-the-wallpaper}}

Everyone is allowed to modify the wallpaper. The modified wallpaper must
be either kept secret, or licensed under a public copyright license such
as the RespTech Wallpaper Public License or the CC-compatible licenses.

\hypertarget{redistributing-the-original-wallpaper}{%
\subsubsection{Redistributing the original
wallpaper}\label{redistributing-the-original-wallpaper}}

The wallpaper can be copied and redistributed in any form. It can be
raster graphic or printed on a physical object, such as a piece of
paper. When redistributing the original wallpaper to the public, the
name of the author and a link to a full copy of this license must be
explicitly specified.

\hypertarget{modified-wallpaper}{%
\subsubsection{Modified wallpaper}\label{modified-wallpaper}}

Modified version of a wallpaper licensed under this license must be
either secret or licensed under a public license, as described in
section 2 above. Thus, the terms and conditions applied for the modified
wallpaper depend on the license that the modified wallpaper is licensed
under, so we cannot provide the terms and conditions for the modified
wallpaper. One exception is, if the modified wallpaper is licensed under
the RespTech Wallpaper Public License, the terms and conditions of this
license take effect, and the terms and conditions for redistributing the
(modified) wallpaper are the same as those stated in section 3 of this
license.

\hypertarget{how-a-rtwpl-licensed-wallpaper-must-be-provided}{%
\subsubsection{How a RTWPL-licensed wallpaper must be
provided}\label{how-a-rtwpl-licensed-wallpaper-must-be-provided}}

The wallpaper must be available in the public. Often times, the author
put their name as part of the content of the wallpaper (e.g.~in the low
corner of the wallpaper). Most (if not all) of the time, the name of the
author is there only for (public) recognition, and is not part of the
content of the wallpaper. If the name of the author is genuinely there
as part of the wallpaper's content, let it be, otherwise, when provided
to the public, the name must be removed from the content of the
wallpaper.

\hypertarget{source-code-of-the-wallpaper}{%
\subsubsection{Source code of the
wallpaper}\label{source-code-of-the-wallpaper}}

Wallpapers are usually created using an image editor, but not all
wallpapers are created that way. Some people write scripts that can
generate wallpapers. Some wallpapers are created using an image editor,
but use a file type that actually makes the wallpaper available in some
kind of human-readable code (e.g.~SVG). Under the effect of the terms
and conditions of the RespTech Wallpaper Public License, the source code
of the wallpaper (e.g.~SVG) or the source code of the script used to
generate the wallpaper does NOT have to be publicly available, however
it is encouraged to make the source code available publicly as well so
people can easily modify the wallpaper.

\hypertarget{limitations}{%
\subsubsection{Limitations}\label{limitations}}

No liability and warranty are held concerning the wallpaper. If claimed
by the provider, the wallpaper must be provided as-is.

\hypertarget{how-to-apply-this-license-to-your-wallpaper}{%
\subsubsection{How to apply this license to your
wallpaper}\label{how-to-apply-this-license-to-your-wallpaper}}

If you wish to license your wallpaper under the RespTech Wallpaper
Public License, chances are your wallpaper is in digital form
(e.g.~raster graphic image). Hence, you are likely going to provide it
on a digital platform. If so, you can do one of the following:

\begin{itemize}
\tightlist
\item
  Explicitly point out that the wallpaper is licensed under the RespTech
  Wallpaper Public License, and provide a link to a copy of the license.
\item
  If you are providing an archive that contains many wallpapers, and at
  least one of them is licensed under the RespTech Wallpaper Public
  License, then put the full license text of this license in a text
  file, typically called ``LICENSE'' or ``COPYING'', and explicitly
  point out which wallpaper(s) is/are licensed under this license.
\item
  Explicitly point out that the wallpaper is licensed under the RespTech
  Wallpaper Public License, along with the following text snippet:
\end{itemize}

\begin{verbatim}
Everyone is allowed to set this wallpaper as wallpaper on any
device. Modified wallpapers can be created from this wallpaper, as
long as they are kept secret for private use or are licensed under
a public license. Redistributing the exact copy of this wallpaper
is allowed, as long as the authors' names and a link to this license
are provided along with the redistribution.
\end{verbatim}

\end{document}
